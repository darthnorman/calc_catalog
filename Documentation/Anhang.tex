% !TEX root = Projektdokumentation.tex
\section{Anhang}
\subsection{Detaillierte Zeitplanung}
\label{app:Zeitplanung}
\tabelleAnhang{ZeitplanungKomplett}
\clearpage

\subsection{Verwendete Ressourcen}
\label{app:Verwendete Ressourcen}

\subsection*{Hardware}
\begin{itemize}
	\item Büroarbeitsplatz mit Desktop-PC
\end{itemize}

\subsection*{Software}
\begin{itemize}
	\item Windows 10 Pro 64\,bit mit Service Pack 1 -- Betriebssystem
	\item Eclipse Mars mit TeXlipse -- Entwicklungsumgebung mit Erweiterung für
	\LaTeX
	\item \acs{XAMPP} -- Apache-Distribution, die MariaDB, PHP und Perl enthält
	\item MariaDB -- Datenbanksystem
	\item phpMyAdmin -- Datenbankmanagementsystem
	\item phpDocumentor -- Software zur Erstellung von Entwicklerdokumentationen
	\item git -- dezentrale Versionsverwaltung
	\item MiKTeX -- Distribution des Textsatzsystems \TeX
	\item Gimp -- Bildbearbeitungsprogramm
	\item Bootstrap -- Framework für Oberflächen 
	\item jQuery -- JavaScript-Bibliothek
	\item List.js -- Bibliothek zum Sortieren und Filtern von Listen im \acs{FE}
	\item Balsamiq -- Programm zur Erstellung von Wireframes
	\item Mozilla Firefox -- Internetbrowser zum Betrachten von Webanwendungen
\end{itemize}

\subsection*{Personal}
\begin{itemize}
	\item Projektmanagerin -- Festlegung der Anforderungen und Abnahme  
	\item Entwickler -- Umsetzung des Projektes
	\item Anwendungsentwickler -- Beratung
\end{itemize}
\clearpage

\subsection{Ausschnitt einer Excel-Tabelle}
\label{app:Exceltabelle}
\begin{figure}[htb]
\centering
\includegraphicsKeepAspectRatio{istanalyse.png}{1}
\caption{Ausschnitt einer Excel-Tabelle einer Kalkulation}
\end{figure}
\clearpage

\subsection{Datenbankmodell}
\label{app:Datenbankmodell}
\begin{figure}[htb]
\centering
\includegraphicsKeepAspectRatio{datenbankschema.png}{1}
\caption{Datenbankmodell}
\end{figure}
\clearpage

\subsection{SQL - Auszug}
\label{app:SQL}
\begin{figure}[htb]
\begin{lstlisting}
CREATE TABLE `item` (
  `id` int(6) unsigned NOT NULL auto_increment,
  `name` varchar(128) collate utf8_unicode_ci NOT NULL,
  `description` text(1000),
  `tmin` float(6,3) unsigned NOT NULL DEFAULT '0',
  `tmax` float(6,3) unsigned NOT NULL DEFAULT '0',
  `category` int(6) unsigned NOT NULL,
  PRIMARY KEY (`id`),
  CONSTRAINT `fk_category` FOREIGN KEY (`category`) REFERENCES category(`id`)
) ENGINE=InnoDB  DEFAULT CHARSET=utf8 COLLATE=utf8_unicode_ci AUTO_INCREMENT=8;

INSERT INTO `item` VALUES(1, 'Berechtigungen vergeben', 'Den Nutzern werden
Rechte vergeben, damit sie ihre Aufgaben erledigen koennen.', 1.25, 2, 1);
\end{lstlisting}
\caption{Auszug aus der Datenbank in SQL-Format}
\end{figure}
\clearpage

\subsection{Oberflächenentwürfe}
\label{app:Entwuerfe}
\begin{figure}[htb]
\centering
\includegraphicsKeepAspectRatio{MockupModules.pdf}{0.7}
\caption{Liste der Module mit Filtermöglichkeiten}
\end{figure}

\begin{figure}[htb]
\centering
\includegraphicsKeepAspectRatio{MockupModul.pdf}{0.7}
\caption{Anzeige der Übersichtsseite einzelner Module}
\end{figure}

\begin{figure}[htb]
\centering
\includegraphicsKeepAspectRatio{MockupTag.pdf}{0.7}
\caption{Anzeige und Filterung der Module nach Tags}
\end{figure}

\clearpage
\subsection{Screenshots der Anwendung}
\label{Screenshots}
\begin{figure}[htb]
\centering
\includegraphicsKeepAspectRatio{tagliste.pdf}{1}
\caption{Anzeige und Filterung der Module nach Tags}
\end{figure}
\clearpage
\begin{figure}[htb]
\centering
\includegraphicsKeepAspectRatio{modulliste.pdf}{1}
\caption{Liste der Module mit Filtermöglichkeiten}
\end{figure}
\clearpage

\clearpage
