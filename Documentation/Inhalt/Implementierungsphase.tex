% !TEX root = ../Projektdokumentation.tex
\section{Implementierungsphase} 
\label{sec:Implementierungsphase}

\subsection{Implementierung der Datenstrukturen}
\label{sec:ImplementierungDatenstrukturen}
Die Datenbank wurde von Hand in Form einer \acs{SQL}-Datei angelegt. Sie wird
per phpMyAdmin in das Datenbanksystem importiert und enthält für jeden Bereich
eine kleine Anzahl an Beispieldaten. Ein Auszug der Datei liegt im
\Anhang{app:SQL}.


\subsection{Implementierung der Benutzeroberfläche}
\label{sec:ImplementierungBenutzeroberflaeche}
Bei der Implementierung der Benutzeroberfläche wurde darauf geachtet, dass eine
Übersichtliche Struktur bewart bleibt. Farben sollten sehr sparsam und nur an
Stellen eingesetzt werden, wo sie für den Nutzer eine sinnvolle Hilfestellung
darstellen. Dies macht sich vor allem bei der Darstellung von Buttons
bemerkbar. Sie stellen die primären Interaktionspunkte für Anwender dar.

Die Navigation, die es ermöglicht zwischen den Listenansichten der Komponenten
zu wechseln, ist ständig prominent am oberen Bildschirmrand platziert. Von dort
aus kann jederzeit auf eine andere Komponente der Anwendung zugegriffen werden.
Die Einstellungen (Firmenname und Mehrwertsteuersatz) werden nur extrem selten
gebraucht und wurden deshalb an den rechten Rand der Navigation gesetzt.

Ein großes Suchfeld über jeder Listenansicht bekommt bei einem Neuladen der
Seite immer den Fokus, sodass direkt mit der Eingabe und damit einer Filterung
begonnen werden kann, ohne dabei mit der Maus erst ins das Feld navigieren zu
müssen. Das Leeren des Feldes funktioniert über das Drücken der
\fbox{ESC}-Taste. Damit ist die Filterfunktion äußerst schnell rein über die
Tastatur bedienbar. Durch zusätzliche Platzhalter in den Eingabefeldern, wird
der Nutzer auf das Format seiner Eingaben hingewiesen.

Screenshots der Anwendung in der Entwicklungsphase mit Dummy-Daten befinden sich
im \Anhang{Screenshots}.


\subsection{Implementierung der Geschäftslogik}
\label{sec:ImplementierungGeschaeftslogik}
Um die Logik den Anforderungen entsprechend abzubilden, wurde zunächst eine
Konfigurationsdatei angelegt, die die Zugangsdaten für die Datenbank enthält.
Die \textit{PHP Data Objects}-Erweiterung stellt die zentrale Schnittstelle
zur Datenbank bereit und wird für alle Interaktionen mit ihr genutzt.

Für die Ausgabe wurde im nächsten Schritt die Funktion \texttt{render()}
geschrieben, die anhand der aufgerufenden \acs{URL} den Controller und den View
bestimmt. Ein Controller-Array beinhaltet alle erlaubten Kombinationen und
vergleicht diese mit der übergebenen URL, dargestellt in
\ref{fig:getCompletePriceMin}: \nameref{fig:getCompletePriceMin}.

Die einzelnen Controller bestimmen mittels des gewählten Views, welche Daten
geholt und durch welches Template diese ausgegeben werden. Das Auslesen der
Daten wird durch die Model-Klassen realisiert, indem Datenbankabfragen getätigt
werden und die Objekte an das Template übergeben werden.

Nach und nach wurden die Methoden zur Bearbeitung, Speicherung und Erstellung
neuer Datensätze geschaffen. Helferfunktionen, die kleine Aufgaben bei der
Darstellung von Werten im Template oder innerhalb von Methoden übernehmen,
wurden gesondert in der Datei \texttt{functions.php} gelistet und global
eingebunden.

Am Beispiel der Klasse \texttt{Calculation} soll die Methode
\texttt{getCompletePriceMin()} vorgestellt werden. Die Methode gibt den
minimalen Gesamtpreis (ohne Steuern) der übergebenen Kalkulation zurück.

\subsection*{\texttt{getCompletePriceMin()}-Methode}
\label{fig:getCompletePriceMin}
\begin{figure}[htb]
\begin{lstlisting}
/**
 * Berechnet den minimalen Gesamtpreis einer Kalkulation
 * @param $id ID der aktuellen Kalkulation
 * @return $pt minimaler Gesamtpreis in Euro
 */
public static function getCompletePriceMin($id) {
	// hole alle Positionen dieser Kalkulation
	$items = self::getItems($id);
	// hole aktuelle Kalkulation
	$currentCalculation = self::show($id);
	
	// Iteriere ueber alle Positionen und berechne
	// den minimalen Gesamtaufwand 
	$pt = 0;
	for ($i = 0; $i < count($items); $i++) {
		$task = $items[$i]->tmin;
		$pt = $pt + $task;
	};
	
	// Tagessatz des Teams
	$team = floatval($currentCalculation->price_team);
	
	// Multipliziere 10% das Gesamtaufwands
	// mit dem Tagessatz der PMs
	$pm = floatval($currentCalculation->price_pm);
	$pm = $pt * 0.1 * $pm;
	
	// Multipliziere die Gesamtzahl der Personentage
	// mit dem Tagessatz des Teams
	$pt = $pt * $team;
	
	// Addiere PM-Aufwand auf Gesamtaufwand
	$pt = $pt + $pm;
	
	// Gib minimalen Gesamtpreis der Kalkulation zurueck
	return $pt;
}
\end{lstlisting}
\caption{\texttt{getCompletePriceMin()}-Methode}
\end{figure}

Im Template \texttt{showCalculation.php} wird diese Methode wie folgt
aufgerufen.
\begin{lstlisting}
<?php echo getBrutto($calculation->getCompletePriceMin($calculation->id)) ?>
\end{lstlisting}
Die Funktion \texttt{getBrutto()} ist eine Hilfsfunktion und daher nicht Teil
der Klasse, sie holt sich den Mehrwertsteuersatz aus den
Einstellungen und berechnet dazu einen Wert inklusive der Steuer.