% !TEX root = ../Projektdokumentation.tex
\section{Implementierungsphase} 
\label{sec:Implementierungsphase}

\subsection{Implementierung der Datenstrukturen}
\label{sec:ImplementierungDatenstrukturen}
Die Datenbank wurde von Hand in einer \acs{SQL}-Datei angelegt. Sie wird per
phpMyAdmin in das Datenbanksystem importiert und enthält für jeden Bereich eine
kleine Anzahl an Beispieldaten. Ein Auszug der Datei liegt im
\Anhang{app:SQL}.


\subsection{Implementierung der Benutzeroberfläche}
\label{sec:ImplementierungBenutzeroberflaeche}

- Filterung/Suche mit Focus
- Liste und Einzelansicht
- Auswahl in Navigation
- Einheitliche Gestaltung
- dezente Farbgebung
- Placeholder

Screenshots der Anwendung in der Entwicklungsphase mit Dummy-Daten befinden sich
im \Anhang{Screenshots}.


\subsection{Implementierung der Geschäftslogik}
\label{sec:ImplementierungGeschaeftslogik}

\begin{itemize}
	\item Beschreibung des Vorgehens bei der Umsetzung/Programmierung der entworfenen Anwendung.
	\item \Ggfs interessante Funktionen/Algorithmen im Detail vorstellen, verwendete Entwurfsmuster zeigen.
	\item Quelltextbeispiele zeigen.
	\item Hinweis: Wie in Kapitel~\ref{sec:Einleitung}: \nameref{sec:Einleitung} zitiert, wird nicht ein lauffähiges Programm bewertet, sondern die Projektdurchführung. Dennoch würde ich immer Quelltextausschnitte zeigen, da sonst Zweifel an der tatsächlichen Leistung des Prüflings aufkommen können.
\end{itemize}

\paragraph{Beispiel}
Die Klasse \texttt{Com\-par\-ed\-Na\-tu\-ral\-Mo\-dule\-In\-for\-ma\-tion} findet sich im \Anhang{app:CNMI}.
