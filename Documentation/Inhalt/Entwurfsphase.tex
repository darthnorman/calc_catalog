% !TEX root = ../Projektdokumentation.tex
\section{Entwurfsphase} 
\label{sec:Entwurfsphase}

\subsection{Zielplattform}
\label{sec:Zielplattform}
Das Programm sollte als eigenständige Anwendung zum Einsatz kommen. Da bisher
keine Sammlung von Kalkulationspositionen vorhanden ist, muss eine Datenbank
eingerichtet werden. Der Einsatz des My\acs{SQL}-Fork MariaDB bewährt sich
täglich im Unternehmen. \acs{PHP} dagegen ist eine Skriptsprache, die
besonders geeignet ist, um eine dynamische Webanwendung auf Basis freier
Software umzusetzen. Sie ist die am häufigsten verwendete Programmiersprache zum
Erstellen von Websites.\citet{PHP} Die Mitarbeiter sind mit den Technologien
vertraut und können gegebenfalls das System warten und weiterentwickeln.

\subsection{Architekturdesign}
\label{sec:Architekturdesign}
Der Kalkulationskatalog soll auf Basis einer \acs{MVC}-Architektur umgesetzt
werden. Diese teilt jede Softwarekomponente einen der drei
Bestandteile -- Datenmodell (engl. model), Präsentation (engl. view) oder
Programmsteuerung (engl. controller) \citet{ModelViewController} -- zu.
Jeder Teil hat einen speziellen Aufgabenbereich, der von denen der anderen
weitestgehend unabhängig ist. 

Das Model repräsentiert die Daten, der View stellt die Daten dar und der
Controller steuert den Ablauf der Anwendung -- das Bindeglied zwischen Model und
View. Durch die lose Verbindung der Komponenten, erhöht sich die
Wiederverwendbarkeit und Austauschbarkeit. So lässt sich beispielsweise die
Oberfläche komplett verändern, ohne dabei das Model anpassen zu müssen. Die
Trennung bevorteilt außerdem das Testen und Warten der Komponenten.

Es wurde sich bewusst für die Erstellung eines eigenen kleinen Frameworks
entschieden. Zum einen, weil der Lerneffekt dadurch am größten ist, zum anderen,
weil jedes vorhandene Framework eine Unmenge an nicht genutzen Code mitgebracht
hätte. Durch ein eigens entwickeltes Framework bleibt der Code schlank und
übersichtlich.

\subsection{Entwurf der Benutzeroberfläche}
\label{sec:Benutzeroberflaeche}
Eine strukturierte Oberfläche ist für eine einfache und zielgerichtete Bedienung
von großer Bedeutung. Erste Wireframes wurden mit dem Tool
Balsamiq \citet{Balsamiq} erstellt und geben die Richtung für den Aufbau der
Overfläche vor. Die Entwürfe befinden sich im \Anhang{app:Entwuerfe}.

Um schnell eine saubere,
responsive\footnote{\url{https://de.wikipedia.org/wiki/Responsive_Design}}
Weboberfläche zu generieren, wurde auf das Framework Bootstrap \citet{Bootstrap}
(ehemals \textit{Twitter Bootstrap}) gesetzt. Als \acs{CSS}-Framework bietet es
Vorlagen für alle erdenklichen Module: Textstyling, Formulare, Listen und Tabellen sind
einige wenige Vorlagen, die in diesem Projekt Verwendung finden.


\subsection{Datenmodell}
\label{sec:Datenmodell}
Das Datenmodell mit eingezeichneten Kardinalitäten ist im
\Anhang{app:Datenbankmodell} einzusehen. Im Zentrum stehen die Kalkulationen,
die jeweils einen Kunden und einen Status besitzen. Kalkulationen bestehen
außerdem aus einer Vielzahl von Positionen. Diese wiederum besitzen eine
Kategorie. Die n:m-Beziehung zwischen Kalkulationen und Positionen wird
durch eine separate Tabelle abgebildet. Globale Einstellungen, wie
Firmenname und Mehrwertsteuersatz, werden in einem gesonderten Tabelle
(\texttt{Company}) abgelegt.

\subsection{Geschäftslogik}
\label{sec:Geschaeftslogik}
Jedes Objekt innerhalb der Anwendung ist eine Instanz einer Model-Klasse. Daraus
ergeben sich folgende Klassen: \textbf{\texttt{Calculation}},
\textbf{\texttt{Category}}, \textbf{\texttt{Company}},
\textbf{\texttt{Customer}}, \textbf{\texttt{Item}} und \textbf{\texttt{Status}}.
Mit der Ausnahme von \textbf{\texttt{Company}} bekommt jede der Klassen eine
Methode \textbf{\texttt{show()}} (Anzeigen) und \textbf{\texttt{create()}}
(Anlegen), sowie \textbf{\texttt{delete()}} (Löschen), \textbf{\texttt{edit()}}
(Bearbeiten) und ggf. \textbf{\texttt{all()}} (Listenausgabe).
Weiterhin gibt es Methoden, um an Informationen anderer Tabellen zu gelangen,
\textbf{\texttt{Calculation::\-getCustomer()}} \zB liefert das Objekt des
Kunden, der der aktuellen Kalkulation zugeordnet ist. Analog dazu sind alle
anderen Methoden aufgebaut.

\subsection{Maßnahmen zur Qualitätssicherung}
\label{sec:Qualitaetssicherung}
Um die Funktionalität der Anwendung zu gewährleisten, wurden Anwendertests
geschaffen. Diese sehen die systematischen Tests aller Funktionen vor, indem
alle Aktionen der Software, mit unterschiedlichen Eingaben ausgeführt
werden.

Wird ein Test nicht bestanden, kann in git ein Fehlerbericht angelegt
werden.\footnote{\url{https://github.com/darthnorman/calc_catalog/issues}} Git
wird außerdem benutzt, um jederzeit einen stabilen Entwicklungsstrang der
Anwendung zu haben, jeder kann sich an der Weiterentwicklung beteiligen und
Vorschläge zur Verbesserung einbringen.
