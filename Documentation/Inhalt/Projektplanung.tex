% !TEX root = ../Projektdokumentation.tex
\section{Projektplanung} 
\label{sec:Projektplanung}

\subsection{Projektphasen}
\label{sec:Projektphasen}
Eine grobe Zeitplanung wurde im Vorfeld erstellt, dabei war sich an der Vorgabe
von 70 Stunden zu halten. Tabelle~\ref{tab:Zeitplanung} stellt den
groben zeitlichen Ablauf des Projektes dar. Die Entwicklung des Katalogs begann
Mitte April 2016 und zog sich über sieben Wochen hin, in denen regelmäßig
ca. 12\,Stunden pro Woche am Projekt gearbeitet wurde.

\tabelle{Zeitplanung}{tab:Zeitplanung}{ZeitplanungKurz}\\
Die detaillierte Zeitplanung findet sich im \Anhang{app:Zeitplanung}.

\subsection{Ressourcenplanung}
\label{sec:Ressourcenplanung}
Alle im Projekt verwendeten Ressourcen finden sich in einer Übersicht im
\Anhang{app:Verwendete Ressourcen}. Benötigte Hardware wurde von \acs{DMK} zur
Verfügung gestellt. Bei der Auswahl der Software wurde darauf geachtet, dass die
Produkte kostenfrei (\zB Open Source) erhätlich und nutzbar sind oder -- wie im Fall
von Windows -- das Unternehmen bereits Lizenzen dafür besitzt.

\subsection{Entwicklungsprozess}
\label{sec:Entwicklungsprozess}
Bevor es in die Umsetzungsphase gehen konnte, war es nötig sich für einen
passenden Entwicklungsprozess zu entscheiden. Dabei galt es zwischen einem
Wasserfallmodell oder einem agilen Prozess abzuwägen. Um eine stetige
Rückmeldung von den Testern zu erhalten, viel die Wahl schnell auf ein agiles
Vorgehen. Dabei wurde aber nicht explizit ein Vorgehen nach Scrum favorisiert,
sondern eine flexiblere Variante, die regelmäßige Treffen in Form von
Testsessions vorsah. Klassische Daily Stand-Ups oder Sprintplanungen fanden
daher nicht statt.

Die strikte Einteilung in Phasen, wie sie in~\ref{sec:Projektphasen}:
\nameref{sec:Projektphasen} vorgenommen wurde, musste entsprechend aufgeweicht
werden, da mehrere Phasen zyklisch auftraten und sich mit geringerem Aufwand
wiederholten.

Agiles Vorgehen hat den Vorteil, dass sehr schnell ein arbeitsfähiger Stand
präsentiert werden konnte. Am Ende jeder Woche wurde eine kurze Vorstellung des
Projektes durchgeführt. Anmerkungen und Fehler wurden notiert und im nächsten
Sprint bearbeitet, sodass eine kontinuierliche Verbesserung des Programms
stattgefinden konnte.