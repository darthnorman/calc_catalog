% !TEX root = ../Projektdokumentation.tex
\section{Fazit} 
\label{sec:Fazit}

\subsection{Soll-/Ist-Vergleich}
\label{sec:SollIstVergleich}
Das Projektziel, einen Katalog mit bereits geschätzten Positionen zu schaffen,
war erfolgreich. Auf lange Sicht muss sich die Anwendung aber erst noch
beweisen. Die Mitarbeiter sind bislang sehr zufrieden und wollen sie
auch weiterhin für ihre Kalkulationen verwenden.

Die Projektplanung wurde im Großen und Ganzen eingehalten. Anweichungen gab es
im Bereich der Umsetzung der Geschäftslogik. Besonders die Methoden zum
Bearbeiten und Hinzufügen neuer Objekte erwiesen sich als schwieriger als
zunächst angenommen. Demzufolge musste an dieser Stelle mehr Aufwand investiert
werden. Im Gegensatz dazu konnte bei der Umsetzung der Öberfläche Zeit gespart
werden, das lag unter anderem an den bereits vorhandenen Erfahrungen mit dem
Framework Bootstrap.


\subsection{Lessons Learned}
\label{sec:LessonsLearned}
Während der Bearbeitung könnte der Autor wertvolle Erfahrungen in
verschiedenen Bereichen der Softwareentwicklung machen. Zwar war mit dem
Schreiben eines eigenen \acs{MVC}-Frameworks ein nicht unerheblicher Zeitaufwand
verbunden, jedoch war der Lerneffekt dabei umso größer. Das Wissen \bzgl des
Umgangs mit \LaTeX, und phpDocumentor konnte ebenfalls ausgebaut und gefestigt
werden.


\subsection{Ausblick}
\label{sec:Ausblick}
Der Katalog läuft stabil und kann im Projektalltag eingesetzt werden.
Erweiterungen sind natürlich trotzdem wünschenswert. Ein wichtiger Punkt für die
Zukunft wäre die in Kapitel~\ref{sec:Einfuehrungsphase}:
\nameref{sec:Einfuehrungsphase} bereits erwähnte Migration bestehender Daten, um
den Einstieg zu erleichtern.

Darüber hinaus gibt es derzeit keine Möglichkeit, Positionen für eine
bestimmte Kalkulation anzupassen. Weitere Felder, wie \bspw
Multiplikatoren für Positionen, könnten nützliche Ergänzungen darstellen.
Auch eine Einstellmöglichkeit für den Prozentsatz von Projektmanagerleistungen,
die auf eine Kalkulation aufgeschlagen werden, fehlt bislang. Gleiches gilt für
eine Rabattfunktion. Die modulare Struktur des Programms steht diesen Neuerungen
nicht im Weg.

Wenn möglich, sollten nach Abschluss eines Projektes die kalkulierten Aufgaben
mit den tatsächlich benötigten Zeiten verglichen werden. Bei Abweichungen können
die geschätzten Zeiten im Programm angepasst werden, damit zukünftige
Schätzungen noch genauer und zuverlässiger werden.