% !TEX root = ../Projektdokumentation.tex
\section{Fazit} 
\label{sec:Fazit}

\subsection{Soll-/Ist-Vergleich}
\label{sec:SollIstVergleich}

\begin{itemize}
	\item Wurde das Projektziel erreicht und wenn nein, warum nicht?
	\item Ist der Auftraggeber mit dem Projektergebnis zufrieden und wenn nein, warum nicht?
	
	\item Wurde die Projektplanung (Zeit, Kosten, Personal, Sachmittel) eingehalten oder haben sich Abweichungen ergeben und wenn ja, warum?
	\item Hinweis: Die Projektplanung muss nicht strikt eingehalten werden. Vielmehr sind Abweichungen sogar als normal anzusehen. Sie müssen nur vernünftig begründet werden (\zB durch Änderungen an den Anforderungen, unter-/überschätzter Aufwand).
\end{itemize}


\subsection{Lessons Learned}
\label{sec:LessonsLearned}
Während der Bearbeitung könnte der Autor wertvolle Erfahrungen in
verschiedenen Bereichen der Softwareentwicklung machen. Zwar war mit dem
Schreiben eines eigenen \acs{MVC}-Frameworks ein nicht unerheblicher Zeitaufwand
verbunden, jedoch war der Lerneffekt dabei umso größer. Das Wissen \bzgl den
Umgang mit \LaTeX, und phpDocumentor konnte ebenfalls ausgebaut und gefestigt
werden.


\subsection{Ausblick}
\label{sec:Ausblick}
Der Katalog läuft stabil und kann im Projektalltag eingesetzt werden.
Erweiterungen sind natürlich trotzdem wünschenswert. Ein wichtiger Punkt für die
Zukunft wäre die in \namref{sec:Einfuehrungsphase} bereits erwähnte Migration
bestehnder Daten, um den Einstieg zu erleichtern.

Darüberhinaus gibt es \zB derzeit keine Möglichkeit, Positionen für eine
bestimmte Kalkulation anzupassen. Darüber hinaus könnten weitere Felder wie \bspw
Multiplikatoren für Positionen interessant sein. Auch eine Einstellmöglichkeit
für den Prozentsatz von Projektmanagerleistungen, die auf eine Kalkulation
aufgeschlagen werden, fehlt bislang. Gleiches gilt für eine Rabattfunktion.
Die modulare Struktur des Programms steht diesen Neuerungen nicht im Weg.

Wenn möglich sollten nach Abschluss eines Projektes die kalkulierten Aufgaben
mit den tatsächlich benötigten Zeiten verglichen werden. Bei Abweichungen können
die geschätzten Zeiten im Programm angepasst werden, damit zukünftige
Schätzungen noch genauer und zuverlässiger werden.