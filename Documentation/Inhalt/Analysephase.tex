% !TEX root = ../Projektdokumentation.tex
\section{Analysephase} 
\label{sec:Analysephase} 

\subsection{Ist-Analyse} 
\label{sec:IstAnalyse}
Um Kalkulationen für öffentliche Ausschreibungen durchzuführen, wird momentan
jeweils eine große Excel-Tabelle angelegt, auf die alle an der Kalkulation
beteiligten Mitarbeiter zugreifen können. In die Tabelle werden
untereinander händisch alle Anforderungen aus dem Lastenheft geschrieben, die es
zu schätzen gilt. Daneben werden die Aufwände eingegeben und am Ende rechnet
die Tabellenkalkulation eine Summe aller Aufwände multipliziert mit
dem Tagessatz des Unternehmens aus. Eine solche Tabelle befindet sich im
\Anhang{app:Exceltabelle}.

Durch eine mündliche Befragung der Mitarbeiter wurde schnell klar, dass keiner
mit der bisherigen Lösung zu 100\,\% zufrieden ist. Die Tabelle wird schnell
unübersichtlich, das Eintragen der Positionen ist zeitaufwendig. Am meisten Zeit
geht aber für das Schätzen selbst verloren: Positionen, die regelmäßig in
Ausschreibungen auftreten, müssen jedesmal neu geschätzt werden. Das führt
nicht selten zu Stresssituationen unter den Beteiligten, die mit anhaltender
Dauer der Schätzung zusehends angespannter werden. Diesen Zustand gilt es um
jeden Preis zu beheben. Eine Sammlung bereits geschätzter Positionen würde die
Arbeit wesentlich beschleunigen und zur psychischen Entlastung beitragen.

\subsection{Wirtschaftlichkeitsanalyse}
\label{sec:Wirtschaftlichkeitsanalyse}

\subsubsection{\gqq{Make or Buy}-Entscheidung}
\label{sec:MakeOrBuyEntscheidung}
Die Mitarbeiter behelfen sich mit der in
\ref{sec:IstAnalyse}\nameref{sec:IstAnalyse} genannten Excel-Tabelle, so gut sie
können. Leider findet sich auf dem Markt für eine derartig spezielle Form der
Schätzung keine passende Software, die allen Vorgaben gerecht wird. Selbst wenn
es Softwarelösungen gäbe, die für diesen Einsatzzweck geeignet sind, wären
anfallende Lizenzkosten ein großes Hindernis. 
Die Alternative, eine eigene Software zu programmieren, stand schon zur
Debatte, wurde aber aus Zeitgründen nie verwirklicht. Deshalb hat es sich
angeboten diese Aufgabe in Form einer betrieblichen Projektarbeit umzusetzen.

\subsubsection{Projektkosten}
\label{sec:Projektkosten}

Die Kosten für die Durchführung des Projekts setzen sich sowohl aus Personal-,
als auch aus Ressourcenkosten zusammen. Laut Tarifvertrag verdient ein
Auszubildender im dritten Lehrjahr pro Monat \eur{1000} Brutto und arbeitet pro
Monat etwa 20 Tage.

\begin{eqnarray*}
8 \mbox{ h/Tag} \cdot 20 \mbox{ Tage/Monat} = 160 \mbox{ h/Monat}\\
\frac{\eur{1000} \mbox{/Monat}}{160 \mbox{ h/Monat}} = \eur{6,25}\mbox{/h}
\end{eqnarray*}

Es ergibt sich also ein Stundenlohn von \eur{6,25}. 
Die Durchführungszeit des Projekts beträgt 70 Stunden. Für die Nutzung von
Ressourcen\footnote{Räumlichkeiten, Arbeitsplatzrechner etc.} beträgt der
Stundensatz pauschal \eur{15}. Für andere Mitarbeiter wird firmenintern
eine pauschaler Stundenlohn von \eur{25} angenommen. Daraus ergibt sich eine
Kostenaufstellung, die sich sich in
Tabelle~\ref{tab:Kostenaufstellung}\nameref{tab:Kostenaufstellung} befindet.
Die Gesamtkosten belaufen sich damit auf \eur{1687,50}.
\tabelle{Kostenaufstellung}{tab:Kostenaufstellung}{Kostenaufstellung.tex}

\subsubsection{Amortisationsdauer}
\label{sec:Amortisationsdauer}
Im Nachfolgenden soll der Zeitpunkt ermittelt werden, bei dem sich die
Entwicklung der Anwendung wirtschaftlich rechnet. Anhand des Wertes kann
beurteilt werden, ob das Projektes aus wirtschaftlicher Sicht sinnvoll ist und
sich zukünftig Kostenvorteile ergeben. Die Amortisationsdauer berechnet sich aus
den Anschaffungskosten geteilt durch Kostenersparnis, die durch die Anwendung
entsteht.

Durch die Einführung eines Katalogs entsteht Sicherheit bei der Schätzung und
der Gesamte Vorgang wird beschleunigt oder Entfällt teilweise ganz.

\paragraph{Beispielrechnung (verkürzt)}
Bei einer Zeiteinsparung von einer Stunde pro Schätzung und Woche für jeden der
drei Mitarbeiter\footnote{an Schätzungen für öffentliche Ausschreibungen nehmen
in der Regel drei Mitarbeiter teil} und 52 Wochen im Jahr, ergibt sich eine
Zeitersparung von:
\begin{eqnarray*}
3 \cdot 1 \mbox{ h/Woche} \cdot 52 \mbox{ Wochen/Jahr} = 156 \mbox{ h/Jahr}
\end{eqnarray*}

Dadurch ergibt sich eine jährliche Einsparung von:
\begin{eqnarray*}
156 \mbox{h} \cdot \eur{(25 + 15)}{\mbox{/h}} = \eur{6240}
\end{eqnarray*}

Die Amortisationszeit beträgt demnach:
\begin{eqnarray*}
\frac{\eur{1687,50}}{\eur{6240}\mbox{/Jahr}} \approx 0,27 \mbox{ Jahre} \approx
14 \mbox{ Wochen}
\end{eqnarray*}

\subsection{Anwendungsfälle}
\label{sec:Anwendungsfaelle}
Im Wesentlichen muss der Kalukulationskatalog zwei Anwendungsfälle abdecken.

Zum einen soll er ans Nachschlagewerk während einer Kalkulation dienen. Dabei
muss die Liste der Positionen gefiltert werden, um eine effektive Suche nach dem
gewünschten Element zu sichern. Dieser Fall ist durch die prominente
Filterfunktion bei jeder Listenansicht gewährleistet.

Zum anderen muss es möglich sein, neue Kalkulationen mit den entsprechend
nötigen Feldern und Positionen anzulegen. Hierbei ist ebenfalls darauf zu
achten, dass der Anwendungsfall mit geringem zeitlichen Aufwand vollzogen werden
kann. Eine Berechnung des Gesamtpreises über alle Aufwände hinweg, wird als
selbstverständlich angesehen.


\subsection{Qualitätsanforderungen}
\label{sec:Qualitaetsanforderungen}
Es wurden keine spezifischen Qualitätsanforderungen an die Anwendung gestellt.
Daher gelten die allgemeinen Anforderungen für Software wie sie in
\citet{ISO9126} formuliert sind.
