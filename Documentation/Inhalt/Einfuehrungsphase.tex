% !TEX root = ../Projektdokumentation.tex
\section{Einführungsphase}
\label{sec:Einfuehrungsphase}
Um die Software zu deployen, muss lediglich ein Webserver mit
einer Datenbank bereitgestellt werden. Die Quelldateien können entweder von
GitHub heruntergeladen oder geklont werden. Damit eine Verbindung zur Datenbank
aufgebaut werden kann, muss ein Duplikat der Datei
\texttt{/Classes/config\_sample.php} in \texttt{config.php} umbenannt werden und
die Verbindungsdaten (Datenbankhost, -Name, -Nutzer und Passwort) müssen
eingetragen sein. Ein Import der beiliegenden Datenbank ist nicht zwingend
erforderlich, die Datensätze können auch aus der Anwendung heraus angelegt
werden.

Eine Migration alter Kalkulationen hat nicht stattgefunden, da es nicht
Bestandteil des Projektes war. Dennoch sollte der Import einiger Bestandsdaten
in naher Zukunft durchgeführt werden, es würde den Einstieg und die initiale
Arbeit mit dem Katalog erheblich erleichtern.
