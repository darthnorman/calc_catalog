% !TEX root = ../Projektdokumentation.tex
\section{Einleitung}
\label{sec:Einleitung}
Die vorliegende Projektdokumentation beschreibt ein IHK-Abschlussprojekt,
welches der Autor im Rahmen der Ausbildung zum Fachinformatiker mit der
Fachrichtung Anwendungsentwicklung erstellt hat. Der Ausbildungsbetrieb, die
Firma \ac{DMK}, ist eine in Chemnitz ansässige Online-Agentur, deren Schwerpunkt
auf der Umsetzung von Webauftritten für mittelständige Unternehmen im privaten
und öffentlichen Bereich liegt. Gegenwärtig beschäftigt \acs{DMK} 38
Mitarbeiter. Der Autor war bereits vor seiner Ausbildungszeit
mehrjähriger Angestellter im Unternehmen und ist hautsächlich mit Arbeiten an
Kundenprojekten im Bereich Front-End betreut. Als zertifizierter
TYPO3-Integrator\footnote{TYPO3 ist ein Open-Source Content Management System,
das besonders im deutschsprachigem Raum Einsatz findet} ist er an der
Einrichtung und Betreuung von Umgebungen des \acs{CMS}' ebenso beteiligt wie an
der Dynamisierung von \acs{HTML}/\acs{CSS}-Prototypen.


Einleitung Problemstellung
Hier sollen Sie Ihr Unternehmen, die für den Ihnen erteilten Auftrag
relevanten Rahmenbedingungen sowie das Ziel dieser Arbeit vorstellen.
Vergessen Sie dabei nicht, dass dies kurz, prägnant und so
erfolgen soll, dass Außenstehende die momentane Situation schnell
und in ihrer Gesamtheit erfassen können.

\subsection{Projektbeschreibung} 
\label{sec:Projektbeschreibung}
Um an öffentlichen Aussschreibungen teilnehmen zu können, muss
\acs{DMK} in Vorleistung gehen, die Aufwände für jede Position des Lastenheftes
schätzen und in einem Angebot festhalten.
Oftmals werden wiederkehrende Positionen für öffentliche Ausschreibungen wieder
und wieder neu geschätzt. Um diesem Vorgehen Einhalt zu gebieten, soll mit der
{\titel} der Prozess der Schätzung verkürzt werden.

\subsection{Projektziel} 
\label{sec:Projektziel}


\subsection{Projektbegründung} 
\label{sec:Projektbegruendung}
\begin{itemize}
	\item Warum ist das Projekt sinnvoll (\zB Kosten- oder Zeitersparnis, weniger Fehler)?
	\item Was ist die Motivation hinter dem Projekt?
\end{itemize}


\subsection{Projektschnittstellen} 
\label{sec:Projektschnittstellen}
\begin{itemize}
	\item Mit welchen anderen Systemen interagiert die Anwendung (technische Schnittstellen)?
	\item Wer genehmigt das Projekt \bzw stellt Mittel zur Verfügung? 
	\item Wer sind die Benutzer der Anwendung?
	\item Wem muss das Ergebnis präsentiert werden?
\end{itemize}


\subsection{Projektabgrenzung} 
\label{sec:Projektabgrenzung}
\begin{itemize}
	\item Was ist explizit nicht Teil des Projekts (\insb bei Teilprojekten)?
\end{itemize}
