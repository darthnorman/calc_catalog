% !TEX root = ../Projektdokumentation.tex
\section{Einleitung}
\label{sec:Einleitung}
Die vorliegende Projektdokumentation beschreibt ein IHK-Abschlussprojekt,
welches der Autor im Rahmen der Ausbildung zum Fachinformatiker mit der
Fachrichtung Anwendungsentwicklung erstellt hat. Der Ausbildungsbetrieb, die
Firma \ac{DMK}, ist eine in Chemnitz ansässige Online-Agentur, deren Schwerpunkt
auf der Umsetzung von Webauftritten für mittelständige Unternehmen im privaten
und öffentlichen Bereich liegt. Gegenwärtig beschäftigt \acs{DMK} 38
Mitarbeiter. Der Autor war bereits vor seiner Ausbildungszeit
mehrjähriger Angestellter im Unternehmen und ist hauptsächlich mit Arbeiten an
Kundenprojekten im Bereich \ac{FE} betreut. Als zertifizierter
TYPO3-Integrator\footnote{TYPO3 ist ein Open-Source Content Management System,
das besonders im deutschsprachigem Raum Einsatz findet} ist er an der
Einrichtung und Betreuung von Umgebungen des \acs{CMS}' ebenso beteiligt wie an
der Dynamisierung von \acs{HTML}/\acs{CSS}-Prototypen.

\subsection{Projektbeschreibung} 
\label{sec:Projektbeschreibung}
Um an öffentlichen Aussschreibungen teilnehmen zu können, muss
\acs{DMK} in Vorleistung gehen, die Aufwände für jede Position des Lastenheftes,
das von der ausschreibenden Organisation kommt, schätzen und in ein Angebot
überführen. Oftmals werden wiederkehrende Positionen für öffentliche
Ausschreibungen immer und immer wieder neu geschätzt. Um diesem Vorgehen Einhalt
zu gebieten, soll mit der {\titel} der Prozess der Schätzung verkürzt werden.

\subsection{Projektziel} 
\label{sec:Projektziel}
Die fertige Anwendung soll einerseits als Nachschlagewerk für bereits
kalukulierte Positionen dienen, um bei neuen Kalkulationen oder Angeboten
schneller eine sichere Aussage über den Aufwand der Aufgaben zu geben.
Andererseits soll sie die Möglichkeit bieten, ganze Kalkulationen möglichst und
bequem zusammenzustellen. Nützlich wäre es zudem, wenn die Aufwände
zusammengerechnet würden, um eine erste grobe Einschätzung des
Projekts zu bekommen und dessen Umsetzbarkeit zu bewerten. Es ist wichtig, dass
dabei ein leichter Einstieg und die einfache Benutzung der Anwendung
gewährleistet wird. Wenn die Hürde zu groß wäre, würde keine Nutzung stattfinden
und das Projekt wäre gescheitert. Eine enge Zusammenarbeit und regelmäßige
Absprachen mit dem Auftraggeber und den späteren Anwendern ist daher
unerlässlich und von Anfang an ein wichtiges Projektziel.

\subsection{Projektbegründung} 
\label{sec:Projektbegruendung}
Die Schätzung von Ausschreibungen ist ein zeitaufwändiger Prozess. Zeit ist, wie
in jedem anderen Unternehmen auch, eine extrem kostbare Ressource in einer
Online-Agentur. Neben der Bearbeitung von Kundenprojekten und organisatorischen
Dingen im Unternehmen, müssen einzelne Entwickler zusätzlich Zeit für
Kalkulationen bekommen, in der keine Entwicklung an Projekten stattfinden kann.
Es geht also wertvolle Entwicklerzeit verloren, weil Kalkulationen für eine
Angebotserstellung eine Vorleistung für die Agentur darstellen. Vergütungen
für die Teilnahme an einer Ausschreibung oder für die Abgabe eines Angebots
finden nur in den seltensten Fällen statt.

Die Einführung eines Katalogs, der wiederkehrende Anforderungen beinhaltet, kann
Ressourcen im Unternehmen freisetzen, wenn dadurch weniger Mitarbeiter über die
gesamte Dauer einer Kalkulation im Prozess involviert sind und die Dauer des
Schätzens verringert wird. Anstatt jeden Punkt für jede Ausschreibung neu zu
schätzen, kann im System nach der Anforderung gesucht werden. Selbst
Projektmanager, die nicht das technische Wissen besitzen, das für eine Schätzung
der meisten Anforderungen nötig ist, können Kalkulationen selbstständig
erstellen.

\subsection{Projektschnittstellen} 
\label{sec:Projektschnittstellen}
Die vorliegende Software, die im Laufe dieser Projektarbeit erstellt wurde, ist
unabhängig von Schnittstellen zu anderen Anwendungen. 

Als Nutzergruppe der neue Anwendung werden alle Mitarbeiter des Unternehmens
gesehen. Dies beinhaltet Projektmanager mit Fokus auf betriebswirtschaftlicher
Ebene sowie Entwickler im \acs{FE}- und \acs{BE}-Bereich.

\subsection{Projektabgrenzung} 
\label{sec:Projektabgrenzung}
Die {\untertitel}
import nicht notwendig, 